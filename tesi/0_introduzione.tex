\documentclass{article}
\usepackage[utf8]{inputenc}

\title{Il Multi-Agent Pathfinding: Importanza e Confronto tra Algoritmi}
\author{Il tuo nome}
\date{}

\begin{document}

\maketitle

\section*{Perché il Multi-Agent Pathfinding è importante?}

Nella società attuale, sempre più sistemi complessi coinvolgono la collaborazione di diversi agenti autonomi. Dai team di robot che lavorano insieme in una fabbrica ai veicoli autonomi che si muovono nelle città, la capacità di questi agenti di coordinare i loro spostamenti in modo efficiente e sicuro è fondamentale. Il Multi-Agent Pathfinding (MAPF) entra in gioco in questo contesto, essendo un settore di ricerca dedicato allo sviluppo di algoritmi che consentano a più agenti di individuare percorsi ottimali all'interno di uno spazio condiviso.

Il MAPF presenta sfide significative se confrontato al Single-Agent Pathfinding (SAPF). Quando più agenti si trovano nello stesso ambiente, diventa importante considerare le interazioni tra loro per evitare collisioni e garantire un movimento fluido. Inoltre, la complessità del problema aumenta ulteriormente in situazioni con vincoli ambientali come ostacoli o zone proibite da superare da parte degli agenti.

Nonostante le difficoltà incontrate, il MAPF offre numerose applicazioni pratiche. Nel campo della robotica, ad esempio, gli algoritmi MAPF efficienti possono essere impiegati per ottimizzare le attività nei magazzini, coordinare squadre di robot soccorritori o gestire droni per la consegna dei pacchi.
Nel campo dei trasporti, il sistema di pianificazione multi-agente può essere impiegato per ottimizzare la circolazione del traffico nelle città, ridurre i tempi di attesa ai semafori e migliorare la sicurezza stradale. Inoltre, il sistema di pianificazione multi-agente è utilizzato anche in simulazioni di folle, videogiochi e situazioni di evacuazione d'emergenza.

\section*{Introduzione ad X*}

Nel campo del Multi-Agent Pathfinding (MAPF), l'algoritmo X* è noto per la sua efficacia e capacità di adattamento a scenari complessi,
 ottenendo il titolo di "Anytime MAPF for Sparse Domain using Window-Based Iterative Repairs".
  Ma che cosa si nasconde dietro questo nome complesso? X* adatta l'algoritmo A* per il pathfinding di
   più agenti che si muovono contemporaneamente, basandosi sull'algoritmo originale sviluppato per un
    singolo agente. L'algoritmo lavora iterativamente, migliorando gradualmente le soluzioni per ridurre i
     conflitti e ottimizzare i percorsi. Durante lo sviluppo di questa tesi, esploreremo nel dettaglio il significato dietro il nome attribuito all'algoritmo X*. 


\section*{Confronto tra X e altri algoritmi di Multi-Agent Pathfinding}

Questa tesi si concentrerà su X* e verrà confrontato con altri algoritmi MAPF popolari, come come CBS (Conflict-Based Search), ICTS (Iterated Conflict-Based Search), ICTS+ID (ICTS con identificazione di conflitti incrementale) e IRC (Informed Reciprocation Reservation).
Studieremo come funzionano i principi di ciascun algoritmo, analizzeremo le loro prestazioni in termini di tempo di esecuzione, qualità del percorso e completezza, e valuteremo la loro idoneità a diversi scenari di MAPF. 

L'obiettivo di questa tesi è quello di fornire una valutazione completa di X* rispetto ad altri algoritmi MAPF, identificando i suoi punti di forza e di debolezza e fornendo indicazioni sulla sua scelta in base alle specifiche esigenze di un'applicazione.

\end{document}
